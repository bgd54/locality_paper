\subsubsection{Airfoil}
Airfoil is a benchmark application, representative of
large industrial Finite Volume CFD applications. It is a non-linear 2D inviscid
airfoil code that uses an unstructured grid and a finite-volume discretisation
to solve the 2D Euler equations using a scalar numerical dissipation. The
algorithm iterates towards the steady state solution, in each iteration using a
control volume approach, meaning the change in the mass of a cell is equal to
the net flux along the four edges of the cell, which requires indirect
connections between cells and edges. Airfoil is implemented using the OP2 domain
specific language \cite{op2}, where two versions exist, one implemented with
OP2's C/C++ API and the other using OP2's Fortran API
\cite{giles2012op2,op2-repo}.

The application consists of five parallel loops: \textbf{save\_soln},
\textbf{adt\_calc}, \textbf{res\_calc}, \textbf{bres\_calc} and \textbf{update}.
In our work we focus on \textbf{res\_calc} since it has indirect increments and
about the 70\% of the time is spent in this kernel on GPUs with global
colouring.  This is the most complex loop with both indirect reads and writes; it
iterates through edges, and computes the flux through them using data accessed indirectly on cells. It is called 2000
times during the total execution of the application and performs about 100
floating-point operations per mesh edge. In each iteration, it reads 5 and
increments 4 double values from each of the 2 indirectly accessed cells, and
reads 2 double values from each of the 2 indirectly accessed nodes.
\\
 The tests are executed on a mesh with 2.8 million cells.

\subsubsection{Volna}
Volna is a shallow water simulation capable of handling the complete life-cycle
of a tsunami (generation, propagation and run-up along the coast)
\cite{dutykh2011volna}. The simulation algorithm works on unstructured
triangular meshes and uses the finite volume method. Volna is written in C/C++
and converted to use the OP2 library\cite{op2}. Volna spends most time in three kernels: \textbf{computeFluxes}, \textbf{SpaceDiscretization}
and \textbf{NumericalFluxes}. We focus on the \textbf{SpaceDiscretization}
kernel that iterates on edges accessing data indirectly on cells, 60\% of total execution time is spent in this kernel. In each iteration,
\textbf{SpaceDiscretization} reads 1 and increments 4 float values from each of
the 2 indirectly accessed cells, and reads 7 float and 1 integer values directly.

Tests are executed in single precision, on a mesh containing 2.4 million
triangular cells, simulating a tsunami run-up to the US pacific coast.

\subsubsection{BookLeaf}
BookLeaf is a 2D unstructured mesh Lagrangian hydrodynamics application from the
UK Mini-App Consortium \cite{uk-mac}. It uses a low order finite element method
with an arbitrary Lagrangian-Eulerian method.  Bookleaf is written entirely in
Fortran 90 and has been ported to use the OP2 API and library. Bookleaf has a
large number of kernels with different access patterns such as indirect
increments similar to increments inside \textbf{res\_calc} in Airfoil. For
testing we used the SOD testcase with a 4 million element mesh. The top three
kernels with the highest runtimes are \textbf{getq\_christiensen1},
\textbf{getacc\_scatter}, \textbf{gather}. Among these (\textbf{getacc\_scatter}) has indirect increments, iterating on cells and incrementing data indirectly on vertices. This kernel reads 17 double values directly, and
increments 4 double values on each of the 4 indirectly accessed nodes in each
iteration.

\subsubsection{Lulesh}\label{sec:lulesh-summary}

\hyphenation{CalcFBHourglassForceForElems IntegrateStressForElems}

Livermore Unstructured Lagrangian Explicit Shock Hydrodynamics (LULESH,
\cite{LULESH2:changes}) represents a typical hydrocode representing the Shock
Hydrodynamics Challenge Problem that was originally defined and implemented by
Lawrence Livermore National Lab as one of five challenge problems in the DARPA
UHPC program and has since become a widely studied proxy application in DOE
co-design efforts for exascale. 

LULESH is a highly simplified application, hard-coded to only solve a simple
Sedov blast problem that has an analytic solution \cite{LULESH:spec} – but
represents the numerical algorithms, data motion, and programming style typical
in scientific C or C++ based applications at the Lawrence Livermore National
Laboratory. LULESH approximates the hydrodynamics equations discretely by
partitioning the spatial problem domain into a collection of volumetric elements
defined by a mesh.

The mesh itself is structured (and generated in the code), but
the algorithm doesn't take this into account and accesses the data through an
eight-dimensional mapping for the hex8 (brick) elements.

We study the \textbf{IntegrateStressForElems} kernel that calculates the
forces in the nodes. Iterating over cells, it reads 3 double values
from each of the 8 indirectly accessed nodes, increments 3 double values for
each node, reads 3 double values directly and writes 1 double value directly. In
our measurements, we used a mesh with \num{4913000} cells and \num{5000211}
nodes.

The original CUDA version of the code contracted this kernel with
\textbf{CalcFBHourglassForceForElems}; the only modifications we did to this
code for our measurements is to remove these parts from the kernel.

\subsubsection{miniAero}\label{sec:mini-aero-summary}

miniAero \cite{miniaero} is a mini-application for the evaulation of programming
models and hardware for next generation platforms from the Mantevo suite
\cite{heroux2009improving}. MiniAero is an explicit (4th order
Runge-Kutta) unstructured finite volume code that solves the compressible
Navier-Stokes equations. Both inviscid and viscous terms are included. The
viscous terms can be optionally included or excluded. For miniAero meshes are
created in code and are simple 3D hex8 meshes. These meshes are generated on the
CPU and then moved to the device. While the meshes generated in code are
structured, the code itself uses unstructured mesh data structures and access
patterns. This mini-application uses the Kokkos library
\cite{CarterEdwards20143202}.

For miniAero we study the \textbf{compute\_face\_flux} kernel that computes the
flux contributions of the faces and increments it with the appropriate cell flux
values. The kernel iterates over the faces of the mesh, and accesses the cells
indirectly. In each iteration, it reads 28 and increments 5 double values from
each of the 2 indirectly accessed cells, and reads 12 double values directly.

The original code (depending on a compile time parameter) either uses the
auxiliary \textbf{apply\_cell\_flux} kernel that does the actual incrementing by
gathering the intermediate results from a large temporary array, or uses atomics
to do it within the kernel. Both the atomics and the work of the auxiliary
kernel was substituted in our code by colouring.

% vim:set et sw=2 ts=2 tw=80:
