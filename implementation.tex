When implementing the library, the goal was to minimise the need for the users
to modify their computation algorithms.

\subsection{User kernels}

The concept of a computation is in the form of a loop (or kernel) body. The same
code can be used as in the serial algorithm, with the restrictions described in
Section \ref{unstructured-meshes} and with the modification that, since the data
layout is not necessarily the usual AoS, the accesses need to use the stride
parameters (supplied to the kernel by the library) that are defined to be the
distance between two consecutive data component (e.g. this is $1$ in the case of
AoS).

Four types of loops are supported: one serial and three parallel. The
parallelisations are done by (1) OpenMP on the CPU, (2) CUDA with global
colouring and (3) CUDA using the shared memory apporach with hierarchical
colouring on the GPU. The CPU versions are not optimised and are only there for
testing and verification purposes.

The user kernel consists of two main levels. In the first, the pointers to the
data arrays are acquired, typecast and the result is written back into global
(GPU or CPU) memory. This is similar in the differently parallelised loops (e.g.
there is usually no difference between the serial and the OpenMP versions), but
there are small variations in the use of GPU shared memory, the synchronisation
steps and the calculation of the loop variable. (By loop variable, we mean the
index from the from-set.) In our current implementation, the user creates this
level from provided templates, but this can be easily automated by means of a
source-to-source translator tool. The second level is the calculation itself
that should be the same for all loop forms.

The data will be automatically copied to the device before the beginning of the
loop when running on the GPU, and the result will be copied back to the host
after. The pointers supplied to the user kernel also point to the location in
the appropriate address space.

The directly accessed data is always in SoA form, while the layout of the
indirectly accessed data is a user supplied compile time parameter. The shared
memory in our kernels is in SoA form, but that is in the hand of the user. The
layout of the mappings is AoS, except when using hierarchical colouring.

The synchronisation is done by the library using multiple kernel launches in the
case of OpenMP and global colouring. The hierarchical colouring has some
additional synchronisation steps, as described by Algorithm \ref{code:shared} in
Section \ref{shared-memory-approach}.

\subsection{Execution planning}

To avoid data races in the parallel loops, we colour the elements to avoid data
races. Two kinds of colouring algorithms is used: global colouring is used for
the OpenMP and the first CUDA parallelisation, while hierarchical or two-layer
colouring is used for the shared memory approach.

The global colouring is a direct generalisation of the greedy graph colouring
algorithm: for all elements of the from-set, its colour will be one of those
that are used but none of its corresponding indirect points have. The function
that chooses from the possible colours is a parameter; in the OpenMP and global
colouring parallelisations, the colour with the least amount of from-set
elements assigned to it (so far) is chosen.

The threads with different colours are started in different kernel launches. The
mappings and the direct accessed data arrays are reordered according to their
colour. This avoids introducing another indirection (which would map from their
index in the current colour to the actual from-set element that it should be
working on).

The hierarchical colouring reorders the threads according to the given partition
and maps the threads to blocks. The block sizes are limited by a parameter
(given by the user), if one block in the partition is larger, it is divided into
two.

The same algorithm described above is then used to colour the thread blocks.
Within the blocks we use heuristic of ordering the threads by removing the one
with the minimum degree, placing it last in the ordering, then recursively
ordering the other threads.

After colouring the threads are sorted according to their colour to reduce warp
divergence, as described in Section \ref{optimisations}.

The mappings and the direct accessed data are transformed into SoA layout to
increase spatial locality. Similarly to the global colouring, these are then
reordered according so they can be directly indexed by the loop variable.

\subsubsection{Reordering}

Before the execution of the loop an optional step is the reordering of from- or
to-set elements using the GPS or the partitioning algorithms described in
Section \ref{increasing-data-reuse}.

We use the Scotch library\cite{scotch} for the GPS algorithm, and the
METIS\cite{metis} library for the multilevel k-way partitioning algorithm (using
64 bit integers for indices). We also tried the recursive bisection algorithm in
the METIS library, but the result was significantly worse, as well as the
partitioning algorithm in the Scotch library, but that failed to stay within
tolerance (and half of the blocks in the partition were empty, others were
larger then requested).

It must be noted that these algorithms were developed for distributing workloads
on computing clusters that typically have much larger block size to total size
ratio. This also caused the reordering (partitioning) phase to be quite long:
even minutes; but this is a one-off cost: the reordering can be saved from the
library and reused many times later. Further improvement could be achieved by
using the parallel versions of the METIS library: ParMETIS\cite{parmetis} and
the alpha version mt-Metis\cite{mtmetis} libraries. Partitioning algorithms
targeting specifically small partition sizes required for CUDA thread blocks are
a target of future research.

% vim:set et sts=2 sw=2 ts=2 tw=80:
